\documentclass{article}
\usepackage{polski}
\usepackage[utf8]{inputenc}
\renewcommand{\baselinestretch}{1.5}
\usepackage{secdot}
\sectiondot{subsection}

\title{Komunikacja wieloprocesowa w języku Linda realizowana przy pomocy pamięci współdzielonej oraz semaforów}
\author{Michał Sypetkowski, Marcin Waszak, Łukasz Wlazły}
\date{}

\begin{document}
	\maketitle
	\newpage

	\section{Treść zadania}
	Komunikacja między-procesowa w Lindzie realizowana jest poprzez wspólną dla wszystkich procesów przestrzeń krotek.
    Krotki są arbitralnymi tablicami dowolnej długości składającymi się z elementów 2 typów podstawowych: \texttt{string}, \texttt{integer}.
    Przykłady krotek: \texttt{(1, "abc", "xyz")}, \texttt{(10, "abc")}.
    Funkcja \texttt{output} umieszcza krotkę w przestrzeni.
    Funkcja \texttt{input} pobiera i atomowo usuwa krotkę z przestrzeni przy czym wybór krotki następuje poprzez dopasowanie wzorca-krotki.
    Wzorzec jest krotką, w której dowolne składniki mogą być niewyspecyfikowane: "\texttt{*}" (podany jest tylko typ) lub zadane warunkiem logicznym. Przyjąć warunki: \texttt{==}, \texttt{<}, \texttt{<=}, \texttt{>}, \texttt{>=}. \\
    Przykład: \texttt{input(integer:1, string:*, string:"xy*")} - pobierze pierwszą krotkę z przykładu wyżej zaś: \texttt{input(integer:>2, string:"abc")} drugą.
    Operacja \texttt{read} działa tak samo jak \texttt{input}, lecz nie usuwa krotki z przestrzeni. Operacje \texttt{read} i \texttt{input} zawsze zwracają jedną krotkę.
    W przypadku, gdy wyspecyfikowana krotka nie istnieje, operacje \texttt{read} i \texttt{input} zawieszają się do czasu pojawienia się oczekiwanej danej.

	\section{Interpretacja treści zadania}
	\subsection{Założenia}
	\begin{enumerate}
		\item \texttt{ELEM\_SIZE}, \texttt{MAX\_TUPLES\_COUNT} - stałe edytowalne przed kompilacją, domyślnie 256 i 4096
		\item Krotki będą przechowywane w obszarze pamięci współdzielonej w postaci listy, do której dostęp będzie miał każdy proces korzystający z API dostępnego w tworzonej bibliotece.
		\item Pamięć współdzielona zawiera listę krotek.
		\item Każdy element listy ma wielkość \texttt{ELEM\_SIZE} bajtów.
		\item Każdy element listy zawiera: dane krotki, wskaźnik na następny element w liście oraz parę semaforów ("pisarz" oraz "czytelni") do synchronizacji.
		\item Wskaźnik na pierwszy element listy jest współdzielony i synchronizowany mechanizmem semaforów.
		\item Nowy element dodawany jest na koniec listy.
		\item Usuwanie elementu z listy wymaga 2 dostępów pisarza (dla 2 sąsiednich elementów - poprzednika i usuwanego).
		\item Odczytywanie polega na przeiterowanie się po liście aż odnajdzie się pasujący element (wymaga dostępów czytelnika dla obecnie czytanego elementu).
		\item Projekt zostanie wykonany w języku C++ (standard C++17).
		\item Pamięć współdzielona oraz semafory zostaną zrealizowane według standardu POSIX.
		\item Maksymalna liczba krotek jest stała - \texttt{MAX\_TUPLES\_COUNT}.
		\item Obszar pamięci na krotki będzie zawierał nagłówek określający ilość elementów krotki i typy jej elementów, a także wartości przesunięć względem początkowego adresu krotki. Wartości elementów będą stanowiły resztę obszaru.
	\end{enumerate}

	\subsection{API}
	Użytkownik biblioteki będzie miał do dyspozycji dwie klasy. Klasa \texttt{Tuple} będzie abstrakcją dla tworzonych krotek i umożliwi łatwiejsze ich tworzenie oraz analizę sytuacji błędnych. Klasa \texttt{Buffer} która będzie udostępniać trzy metody zawarte w treści zadania:
	\begin{itemize}
		\item \texttt{int output(Tuple t) }
		\item \texttt{std::optional<Tuple> input(std::string expr, unsigned int timeout)}
		\item \texttt{std::optional<Tuple> read(std::string expr, unsigned int timeout)}
	\end{itemize}
	Klasa \texttt{Tuple} będzie w stanie przedstawić swoją zawartość w dwojaki sposób: w formie wygodnej dla użytkownika oraz w postaci sformatowej do zapisu w obszarze pamięci współdzielonej.
	Wyrażenia przekazywane przez metody \texttt{input} oraz \texttt{read} zostaną przeanalizowane przez klasę \texttt{ExpressionParser} i posłużą do zbadania zawartości pamięci współdzielonej.

	\subsection{Obsługa błędów}
	Wszystkie sytuacje błędne będą sygnalizowane poprzez wartości zwracane z funkcji.
	Metoda \texttt{output} może zakończyć się na dwa sposoby: sukcesem lub porażką z powodu braku pamięci na nową krótkę. Sytuacje te zostaną zmapowane na wartości liczbowe i zwrócone do użytkownika.
	Metody \texttt{input} oraz \texttt{read} zwrócą krotkę, jeśli istnieje lub nic jeśli krotka pasująca do wyrażenia nie znajdzie się w buforze przez upłynięciem podanego czasu. Klasa \texttt{std::optional} z biblioteki standardowej pozwoli na bezpieczne obsłużenie obu sytuacji.



\end{document}
