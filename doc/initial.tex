% !TeX spellcheck = <none>
\documentclass{article}
\usepackage{polski}
\usepackage[utf8]{inputenc}
\renewcommand{\baselinestretch}{1.5}
\usepackage{secdot}
\sectiondot{subsection}

\title{Komunikacja wieloprocesowa w języku Linda realizowana przy pomocy pamięci współdzielonej oraz semaforów}
\author{Michał Sypetkowski, Marcin Waszak, Łukasz Wlazły}
\date{}

\begin{document}
	\maketitle
	\newpage
	
	\section{Treść zadania}
	TU TRZEBA WPISAĆ TREŚĆ!!!!!!!!!!!!!
	
	\section{Interpretacja treści zadania}
	\subsection{Założenia}
    \texttt{ELEM\_SIZE}, \texttt{MAX\_TUPLES\_COUNT} - stałe edytowalne przed kompilacją, domyślnie 256 i 4096,
	\begin{enumerate}
		\item Krotki będą przechowywane w obszarze pamięci współdzielonej, do której dostęp będzie miał każdy proces korzystający z API dostępnego w tworzonej bibliotece.
        \item Pamięć współdzielona zawiera listę krotek.
        \item Każdy element listy ma wielkość \texttt{ELEM\_SIZE} bajtów.
        \item Każdy element listy zawiera dane krotki, wskaźnik na następny element w liście oraz writer-readers (paczkę semaforów/mutexów) do synchronizacji.
        \item Wskaśnik na pierwszy element listy jest współdzielony i synchronizowany mechanizmem writer-readers.
        \item Dodawanie elementu jest na koniec listy (dla optymalizacji współdzielony jest również synchronizowany mechanizmem writer-readers wskaźnik na koniec).
        \item Usuwanie elementu z listy wymaga 2 dostępów pisarza (dla 2 sąsiednich elementów - poprzednika i usuwanego).
        \item Odczytywanie polega na przeiterowanie się po liście aż odnajdzie się pasujący element (wymaga dostępów czytelnika dla obecnie czytanego elementu).
		\item Synchronizacja między procesami będzie możliwa, dzięki wykorzystaniu semaforów.
		\item Projekt zostanie wykonany w języku C++ (standard C++17).
        \item Maksymalna liczba krotek jest stała - \texttt{MAX\_TUPLES\_COUNT}.
		\item Na początku obszaru pamięci współdzielonej zostaną umieszczone dwie tablice: jedna będzie zawierała mapowanie obszaru pamięci, co pozwoli na szybkie znajdowanie wolnych miejsc na nowe krotki, druga będzie zawierała mapowanie tablicy semaforów i pomoże znaleźć wolny semafor, na którym proces będzie mógł się zawiesić.
		\item Tablica semaforów, która zostanie użyta przez bibliotekę będzie zawierać 256 pól, z czego pierwsze cztery zostaną użyte do synchronizacji dostępu do obszarów współdzielonych, natomiast pozostałe będą mogły służyć procesom do zawieszenia się w czasie oczekiwania na krotkę pasującą do wyrażenia.
		\item W systemie będzie obecny demon serwisowy, którego zadaniem jest inicjalizacja tablicy semaforów oraz obszaru pamięci współdzielonej. Będzie on także wybudzał procesy po pojawieniu się nowej krotki w celu sprawdzenia, czy warunek oczekiwania został spełniony.
		\item Obszar pamięci na krotki zawierał nagłówek określający ilość elementów krotki i typy jej elementów + ich offsety, wartości elementów stanowią resztę obszaru.
	\end{enumerate}

	\subsection{API}
	Użytkownik biblioteki będzie miał do dyspozycji dwie klasy. Klasa \texttt{Tuple} będzie abstrakcją dla tworzonych krotek i umożliwi łatwiejsze ich tworzenie oraz analizę sytuacji błędnych. Klasa \texttt{Buffer} która będzie udostępniać trzy metody zawarte w treści zadania:
	\begin{itemize}
		\item \texttt{int output(Tuple t) }
		\item \texttt{std::optional<Tuple> input(std::string expr, unsigned int timeout)}
		\item \texttt{std::optional<Tuple> read(std::string expr, unsigned int timeout)}
	\end{itemize}
	Klasa \texttt{Tuple} będzie w stanie przedstawić swoją zawartość w dwojaki sposób: w formie wygodnej dla użytkownika oraz w postaci sformatowej do zapisu w obszarze pamięci współdzielonej. 
	Wyrażenia przekazywane przez metody \texttt{input} oraz \texttt{read} zostaną przeanalizowane przez klasę \texttt{ExpressionParser} i posłużą do zbadania zawartości pamięci współdzielonej.
	
	\subsection{Obsługa błędów}
	Wszystkie sytuacje błędne będą sygnalizowane poprzez wartości zwracane z funkcji.
	Metoda \texttt{output} może zakończyć się na dwa sposoby: sukcesem lub porażką z powodu braku pamięci na nową krótkę. Sytuacje te zostaną zmapowane na wartości liczbowe i zwrócone do użytkownika.
	Metody \texttt{input} oraz \texttt{read} zwrócą krotkę, jeśli istnieje lub nic jeśli krotka pasująca do wyrażenia nie znajdzie się w buforze przez upłynięciem podanego czasu. Klasa \texttt{std::optional} z biblioteki standardowej pozwoli na bezpieczne obsłużenie obu sytuacji.
	
	
	
\end{document}
